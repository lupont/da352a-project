Bithoven is an application for the classical music enthusiast, who has been
irritated by the lack of information about the pieces they listen to. Mainstream
music services focus more on popular music, often only consisting of an artist,
a title, and an album. When listening to classical music, you often want to know
who the composer is, who wrote the piece in the first place, eventual soloists,
and more. Bithoven allows for easily finding all of this, through a simple
command-line interface. It is a user-based service, requiring signing in with an
email account. This is because Bithoven supports playlists. Any user can create
as many playlists as they want, consisting of any performances they want. The
user can search for specific albums, composers, content groups, pieces,
performers, and performances, and instantly get access to the information they
want. After finding a performance of interest, the user can choose to play it.
This works by opening a web browser to the performance's internal URL.

NoSQL has, in recent years, become quite attractive for developing applications.
The different kinds of NoSQL databases allow for many different use cases;
simple but powerful query languages, ease of use, good interaction with web
technologies, and of course, the ability to handle big data. Choosing to use
NoSQL therefore leads to another important question: what type should one use?
In some cases, the choice is clear, while in others, it might be more difficult.

In our case, the choice was quite clear from the start. Bithoven is an
application that relies heavily on the relationships between data, and the type
of NoSQL database that handles this the best is graph databases. Therefore, we
used Neo4j to develop our database. Its natural query language \emph{Cypher}
is intuitive and powerful, and while using it during development it became clear
that we made the right choice.

It can be argued that graph databases handle scaling the worst of the types of
NoSQL databases. We realized this from the start, but came to the conclusion
that while it might evolve into a problem in the future, they still fit the best
for our use case. Since Bithoven is read-bound, scaling it is not as big of a
problem as it would have been had the application been more write-bound.

Other types of NoSQL databases include Key-Value, Document, and Column-Based.
These all have their use cases, and we have discussed advantages and
disadvantages of all of them -- in general as well as specifically regarding
Bithoven.
