\subsubsection{Development Process}
Bithoven is developed in Java with Neo4j using the official Neo4j driver. The
driver is managed through Maven, and is the only external dependency of the
project. To make the development process simpler, we used GNU Make to wrap the
most common Maven commands. We used Git in for version control.

\subsubsection{Neo4j}
As described, we chose to use Neo4j as a database for Bithoven. Neo4j is itself
written in Java, and is cross-platform. We found that the simplest way of
installing Neo4j was via Docker, since Neo4j provides an official container.
A thorough guide of how to get Neo4j set up through Docker is provided in
section \ref{inst-instr-inst}.

\subsubsection{Installation}
\label{inst-instr-inst}
In order to install Bithoven, you need the following software installed on your
computer:
\begin{enumerate}
    \item Java
    \item Docker
\end{enumerate}

The main part of installing Bithoven is getting the Neo4j database set up. After
that, it is simply a question of running the provided jar file. The following
will describe how to do this.

\begin{enumerate}
    \item Copy the provided \texttt{neo4j} directory to your home directory.

    \item Install and launch the official Neo4j Docker Image:
\begin{verbatim}
docker run \
 --name testneo4j \
 -p7474:7474 -p7687:7687 \
 -d \
 -v $HOME/neo4j/data:/data \
 -v $HOME/neo4j/logs:/logs \
 -v $HOME/neo4j/import:/var/lib/neo4j/import \
 -v $HOME/neo4j/plugins:/plugins \
 --env NEO4J_AUTH=neo4j/test \
 neo4j:latest
\end{verbatim}
This command gets the latest version of Neo4j from Dockerhub, maps the needed
ports from your host machine to the container, and likewise maps the folder called
\texttt{neo4j} in your home directory to important parts of the container. The
reason for manually adding the directory is that the provided one contains
pre-made data that is needed in order to be able to explore the application.
NOTE:
Bithoven expects the bolt port to be 7687, so do not change this to any other
port.

    \item In a web browser, navigate to \texttt{https://localhost:7474} to check
        that the installation worked. If you are greeted by the Neo4j web
        dashboard, everything is fine -- in which case you can sign in with
        username \emph{neo4j} and password \emph{test}.

    \item Run the following command in the directory of the
        \texttt{bithoven.jar} file:
\begin{verbatim}
java -cp bithoven.jar bithoven.App
\end{verbatim}

    \item You should now be greeted by a login page. Enter an e-mail address and
        a password to create a new user.
\end{enumerate}

Once you have created a user, the application will suggest you to use the
\texttt{help} command to see what actions can be performed. Use this help to
explore the application fully.

\subsubsection{Links}
\label{installation-instructions-links}
\begin{enumerate}
    \item \href{https://neo4j.com/developer/java/}{Neo4j Official Java Driver}
    \item \href{https://maven.apache.org/download.cgi}{Maven}
    \item \href{https://gnu.org/software/make/}{GNU Make}
    \item \href{https://git-scm.com/}{Git}
    \item \href{https://docker.com}{Docker}
\end{enumerate}
