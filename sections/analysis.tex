% 4-7 pages
\subsection{Key-value Databases}
\label{analysis-kv-db}
The main selling point of key-value databases is that they are quite flexible.
Lookups are very fast since all data is connected to a single, unique key. The
values can of course be of any type as well, meaning that one is not tied to a
specific schema -- anything one wants to store can be stored. Of course, this
means that the only way of querying the data is via its (primary) key. Thus, if
for any reason one would need to get the data by some other way, a key-value
database might not be the best choice.

Because of this ``constraint'', any sort of relationship becomes a hassle to
deal with in a key-value database. Since queries are done with the keys only,
connecting two items meant two queries.

The main advantage of using a key-value database for Bithoven is that the
project is quite read-heavy. The only times writes are performed is when a user
creates/deletes/modifier a playlist, and when new data is added by the
administrators. The absolute majority of the time, users would only perform read
operations -- which key-value databases can be quite performant with. However,
as we will soon discuss, this alone does not mean that it is a viable option for
our project.

For Bithoven, a key-value database does not fit well for a couple of main
reasons. The first reason is that our project is quite relationship-heavy. From
a piece (a track), we want to be able to find its composer, its performers,
album, and its performances. This is the case for many of the models, and in
some cases we want to be able to go both ways. Implementing this in a key-value
database would be quite cumbersome, since it would mean a lot of queries to get
the data we want.

\subsection{Document Databases}
\label{analysis-doc-db}

It can be quite natural to think of Bithoven in terms of documents, which is the
case for many database applications -- and one of the reasons why document
databases are so tempting. We have a lot of models, many of them in some way
connected with one another, and being able to fetch a document of this could be
very nice. However, there would be a \emph{lot} of wasted space if each document
should hold its relationships entirely. On the other hand, if the documents
contained references to its related documents, multiple queries
for---potentially---large documents would be needed to get the data the
application requires. 

Much of Bithoven's data storage has more to do with relationships than
properties. For instance, a \emph{playlist} in Bithoven is by itself nothing
more than a title. All the useful data -- i.e. who owns it, what performances it
contains -- is neither a property of the playlist nor the performance. Thus,
playlist documents would only contain a single property. Document databases are
less than ideal for relationship-heavy datasets, which makes it difficult
finding any real advantages of them for our use case.

Bithoven is a read-heavy application indeed, as briefly discussed in section
\ref{analysis-kv-db}. Scaling a document database for reading is often as simple
as adding more read slaves to the replica set. This is quite attractive for
Bithoven. However, it is questionable if it is enough of an attraction to make
document databases worth it. We still have the previously described issues to
worry about, and many other NoSQL databases scale easily enough for reads as
well.

This all leads to the fact that document databases is not the best of fit for
Bithoven, and thus the quest for the perfect NoSQL database type continues.

\subsection{Graph Databases}
\label{analysis-graph-db}

% Cons
One downside of Graph Databases is that the architecture makes it difficult to distribute the database over multiple servers. Since a node can point to any other node, relationships across servers will reduce the performance of the system. All benefits from faster graph traversal will be eliminated when server routing creates a big performance bottleneck.

% Pros
Just like relational databases, a Graph Database is consistent within a single server. Implementations like \emph{Neo4j} is fully ACID compliant. Changes are made through transactions which are either fully successful or completely rolled back. Graph Databases have great performance when traversing a graph and is useful for applications like social graphs, routing, and recommendation engines.

% Our project - benefits: advanced relationships
In our project, there are a lot of different relationships between data items that could benefit from being implemented as a Graph Database. One example is the relationship between a performer and a performance. The same performer can have multiple roles in a performance, for example being both the conductor and the soloist. With a Graph Database, this can be modeled with two relationships between two nodes, each with a different relationship type. In a relational database, this would have to be implemented with an extra table between performer and performance where the type of the relationship could be specified. Another way could also be to have multiple instances of the same performer, but each having a different performer type.

% Our project - benefits: advanced read queries are fast
The most common use cases in our project is different types of data retrieval. This includes searching with a keyword, or a specific constraint. For example, ``List all performances of any piece composed by Brahms''. This type of query would require multiple joins in a relational database system, where the Composer, Piece, and Performance tables would have to be joined and filtered. When using a Graph Database, this type of query could be done using graph traversal instead. This would provide better performance, especially once the tables become large. The relationships would also always be modeled in the database, whereas in a relational system, they would need to be recalculated for every query. Although out of scope for this project, future features that would be very reasonable for this kind of application could include even more constraints where graph traversal would deliver good performance. For example, ``List all performances of Beethovens symphonies conducted by Herbert von Karajan''.

% Our project - downsides: scalability
The biggest downside of using a Graph Database for our project is the difficulty to scale the database. Since the major benefit of a Graph Database is the performance gains of graph traversal, sharding the database and placing different nodes on different servers would most likely eliminate these performance gains. However, there is a possibility to scale the system for better read-access and availability by using multiple read-only slave servers. Our project would really take advantage of this, since most of the data in the database are not subject to frequent change. A \emph{piece} in the database will always be composed by the same composer (unless there is a music history breakthrough) and the piece will always be part of the same content group. This means that consistency would not be a major problem.

% Our project - where changes are made
The part of this project where most changes are made to the database are the changes initiated by the user. This includes creating and deleting playlists, and adding and removing performances to or from a playlist. This type of data is user specific, so quick consistency throughout the database is not a big concern. However, losing the changes would be frustrating for user. But this data loss would not result in lost sales or critical data.

% Our project - sharding the database
A Graph Database could be scaled by sharding the data based on domain-specific knowledge which could be managed in the application. This would probably become a bigger challenge as the database grows, since there are no suitable attribute in our project to divide the data by. All performances, composers, and performers in the database should be available for all users to search for, list by, or put in their playlists.

\subsection{Column Databases}
\label{analysis-col-db}

% Pros - Column-oriented
A Column-Oriented Database makes data processing faster when data from the same column needs to be processed from multiple rows. This is a typical use case in data warehouses and analytics where it is not so common to process all columns of a row at the same time.
Since compression work on localized subsets of data, data stored together by column will often require less computation to acquire a high compression rate. When the data in a column is sorted, even higher compression can be acquired by storing delta values between columns instead of absolute values.

% Cons - Column-oriented
Just as Column-Oriented Databases are advantageous when processing data from the same column, they are disadvantageous when processing data from the same row. The different columns of a row are scattered in storage, and requires retrieval from each column store to assemble the row. This can however be slightly improved by using caching and projections.

% Pros - Column-Family Store
In a Column-Family Store it is easy to add and remove columns from a single row since different rows can have different columns. In a relational database, this would require the schema to change and be harder to do.
Because every node in the cluster are equal, it is very easy to scale a Column-Family Store by simply adding more nodes. This can improve capacity, availability, and read/write performance, depending on how the system is configured.

% Cons - Column-Family Store
The Column-Family store Cassandra does not have transactions, but rather a write is only atomic at the row level. This means that rollbacks must be implemented at the application level, or by using an external transaction system like \emph{ZooKeeper}.
Column-Family stores also makes it hard to create prototypes where queries need to change frequently. When changing the queries, the design of the column family will probably have to change as well.

% Our project - Pros
The data model in this project requires tables where relationships are either optional or unspecified. For example, a performance could have one or one hundred different performers, and a performance could be part of an album and/or a content group. Implementing this in a relational database would require an extra linking table between performance and performer. All performances would need to have columns with an optional foreign key for album and content group, which would sometimes unnecessarily take up space in the database. A Column Database could implement the performance table using a column family for performers. The number of columns could vary for each row depending on how many performers a particular piece has. Other fields -- album and content group -- could be excluded from the performances which does have this, without it taking up extra space in the database.

% Our project - Cons
A downside with using a Column Database is that the columns for a row is often scattered across multiple disks and servers. In this project, it is very common to list all information about a row every time it is accessed. For example when playing a performance, the application should retrieve and display all relevant metadata, e.g. name, length, performers, composers, album, content group, etc. There are not many functions which would benefit from column data being stored contiguously in memory.

% Our project - Pros
A benefit with with using a Column Database for our project is the ease of scaling it. New nodes can easily be added to the database cluster to extend the storage capabilities, as well as availability. Since this project does not require many frequent changes to the database, it can be configured to have very fast read performance and consistent write performance.
