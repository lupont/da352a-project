% 1 page

\subsection{The Non-Relational Concept}
\label{intro-nosql-concept}
% Pontus
The term \emph{non-relational} in the database context refers to a type of
database that does not store its data in the traditionally common tabular
format. This means that it does not store the data in a relational way, which
is implied by the name. The problem with storing data relationally, and thus
the problem NoSQL wants to solve, is that scaling a database is non-trivial.

\subsubsection{Scaling}
\label{intro-scaling}
There are two main ways of scaling a database: \emph{Scale-Up} and
\emph{Scale-Out}. Scale-Up means giving the server more resources, in the form
of memory, storage capacity, processors, etc. This is also referred to as
scaling the database \emph{vertically}. Scale-Out on the other hand, means
adding more identical services -- oftentimes this means more servers -- to the
architecture. This is also referred to as scaling the database
\emph{horizontally}.

Vertical scaling is almost always an option. However, the price-to-performance
increase is often not that high---meaning you have to spend a lot of money for
minor performance increases. This is also the only way to easily scale
relational databases, because they were designed without scale-out in mind.

This is what non-relational databases do differently. The entire idea of them
is to be able to handle big data, which is difficult -- if not impossible --
without being able to scale horizontally in an efficient manner. This is because
of the fact that they do not store the data relationally.

\subsubsection{History}
\label{intro-history}
NoSQL has been around since its introduction by Carlo Strozzi in 1998. It did
not immediately become a hit, but people recognized its pros compared to
the traditional, relational databases, and a lot of effort has since been put
into improving and standardizing NoSQL databases. Many big companies not only
use NoSQL, but have also contributed directly by introducing new types of NoSQL
databases---in 2004, Google launched their BigTable database. In 2007, Amazon
released their paper on Dynamo, in 2008, Facebook open-sourced their Cassandra
NoSQL database, and so on. During this time, the term ``NoSQL'' was not used
very diligently. It was not until 2009 when the term was re-introduced that it
really took off, and nowadays it is used everywhere---especially in web-related
projects, since that is what its current aim is. One of the formal requirements
of a database in order to be allowed to be called NoSQL is that it is
open-source.

The terms NoSQL and Non-Relational are equivalent, both meaning a database that
does not store data relationally. In recent years, however, NoSQL has undergone
a change. At first, it simply meant that the database did not use SQL as its
query language. But, as more and more NoSQL databases showed up, people realized
that since SQL is so good, there should be no need to have to memorize another
syntax for doing the same things. Because of this, many NoSQL databases today
provide query languages influenced by SQL, with the goal of feeling like the
same language. Thus, NoSQL today means ``Not Only SQL'' more than \emph{no} SQL.

There exists many different types of non-relational databases, and in the
following sections we will explore four of the most common types; namely
\textbf{key-value} (\ref{intro-kv-db}), \textbf{document} \ref{intro-doc-db},
\textbf{graph} \ref{intro-graph-db}, and \textbf{column-family stores}
\ref{intro-col-db}.

\subsection{Key-Value Databases}
\label{intro-kv-db}
% Pontus
Key-value databases are the simplest type of NoSQL-database. Its basic storage
structure is a dictionary. This dictionary contains values connected by, as the
name suggests, keys. To get a value, you must go via its key -- there is no
other way. The value can be an integer, a string, an array, or a JSON object.
This makes it very flexible, while being extremely simple to use. 

The three operations a key-value database provides include \texttt{get(key)},
\texttt{put(key, value)}, and \texttt{delete(key)}. With these operations, you
have full control over the database. All handling of more complex logic must
reside in the client code, which can be seen as a positive or a negative thing.
On one hand, it keeps the database very simple to use and learn, but on the
other hand, it might stress the client if you have to execute lots of code when
querying the database. Though if that is the case, a key-value database might
not be the best choice in the first hand -- see the following sections for
alternatives.

Amazon DynamoDB and Riak are two examples of popular Key-Value Databases. These
are different implementations, which means that some things differ greatly
between them. For example, Amazon DynamoDB is cloud-only -- meaning instances
can't be hosted locally -- while Riak can be hosted anywhere. Also, DynamoDB is
not open-source, which effectively goes against the criteria that NoSQL
databases should be open-source. The technology and terms are still quite young,
which could explain why this is the case---a unanimous decision of a formal
specification has not been reached yet.


\subsection{Document Databases}
\label{intro-doc-db}
% Pontus

\subsection{Graph Databases}
\label{intro-graph-db}
% Alex
% Introduction
A Graph Database takes another approach to database design than other databases. Besides storing information about different entities, it also stores information about the relationship between entities. Graph Databases can be said to be graph-oriented and provide certain properties not found in any other type of database.

% Components
Graph Databases are made up of vertices and edges, more commonly referred to as as nodes and relationships. Both nodes and relationships can have different properties. The properties for nodes are the same as in other databases, e.g. name, and address. But unlike other databases, relationships also have properties, e.g. type or strength. Relationships will also have a direction. The relationships are flexible and new ones can be added without needing to change the database schema. Nodes cannot be removed if they have a relationship, i.e. dangling relationships are not allowed.

% Mathematics and computer science
In mathematics, graph theory has existed as a major branch for a long time. The same is true for computer science, which has built upon the graph theory from mathematics for many decades. Graph Databases uses knowledge from these fields and can provide easy and performant graph traversals. They can also provide path finding between two nodes, and use computer science algorithms to e.g. find all paths or the shortest path between to nodes.

% Comparison - relational
Relational databases can also simulate a graph through the use of foreign keys and self joins. But with large tables the performance decreases significantly with each additional join. There is also no way to perform graph traversal where the depth is unknown beforehand, since the exact number of joins has to be specified in the SQL query. A Graph Database will have persistent relationships, but in a relational database the relationships have to be recalculated for every query.

% Comparison - nosql
Other NoSQL solutions does not have the ability to perform efficient graph traversals due to the fact that there are no joins available, so creating relationships is harder. Most often, graph traversal needs to be implemented through logic in the application as it is almost impossible to do inside the database.

% Types
There are two major types of Graph Databases: \emph{Resource Description Framework (RDF)} and \emph{Property Graphs}. RDFs provide information based on triples on the form \texttt{entity: attribute :value}. Property Graphs provide a richer model where nodes and relationships can have attributes.

% Implementations
One of the most popular Graph Databases on the market today is \emph{Neo4j}. It is a Property Graph which support billions of nodes, and ACID compliant transactions. Other available databases are \emph{Infinite Graph}, \emph{OrientDB}, and \emph{FlockDB}.

% Query languages
There exists several query languages for Graph Databases, two of the most well known are \emph{Cypher} and \emph{Gremlin}. Cypher was developed for Neo4j and is a declarative query language, optimized for graph traversals. Gremlin are procedurally oriented and are available for all graph engines that implement \emph{Blueprints Property Graph}, e.g. Neo4j, Titan, and OrientDB.

% Scaling
Although sharding is a common scaling technique for other NoSQL databases, it is not so straightforward for Graph Databases. This is because they are relationship-oriented rather than aggregate-oriented, and every node in the system can be related on every other node. Despite this, a Graph Database system can be scaled in a few different ways. When all data can fit on a single machine, the read-access can be scaled by adding additional read-only slave nodes. This improves availability but requires that all the data fits on a single machine. There is a master server where writes are initiated, and eventually gets pushed to all of the slave servers. Slaves can receive a write request, but will first write the change to the master server, before updating itself. The first machine in a cluster becomes the master, and if the master is taken offline, a new master is chosen from the available slaves. Sharding can be done but, like relational databases, must be done from the application side and will probably be based on domain-specific knowledge.

% Graph Compute Engines
Graph processing algorithms have been incorporated into other database systems through \emph{Graph Compute Engines}. The dataset can be resident in any type of database and the Graph Compute Engine provides capabilities such as graph traversals. This allows for better distribution and scaling, but comes with the cost of not being as performant as a true Graph Database. Common Graph Compute Engines are \emph{Apache Giraph}, \emph{GraphX} and \emph{Titan}.

\subsection{Column Databases}
\label{intro-col-db}
% Alex

% Introduction
A Column Database is oriented around columns rather than rows. This was an idea first used in the 1970s, but was not commercially available until the 1990s.
% Types
There are two types of Column Databases with slightly different architectures and tradeoffs: \emph{Column-Oriented Databases} and \emph{Column-Family Stores}.
% Implementations
Some commonly used column-oriented databases are \emph{Vertica}, \emph{InfoBright}, \emph{VectorWise}, and \emph{MonetDB} and some commonly used Column-Family Stores are \emph{Cassandra}, \emph{Hbase}, \emph{Hypertable}, and \emph{Amazon SimpleDB}.

\subsubsection{Column-Oriented Databases}

% Components
Column-Oriented Databases are organized by column on disk, instead of rows as other databases.
% Comparison
The idea of basing a database around columns have also been implemented in some relational databases today.

% Data warehouse
A data warehouse often use a \emph{star schema} which has a large \emph{central table} with facts that are connected to many smaller \emph{dimension tables}. This allows aggregate queries to execute quickly. A star schema is not fully normalized according to the relational model, but contains redundant data to improve performance.
Columnar databases can be implemented using a delta store with write optimization to support the need for real-time updates in data warehouses. A delta store is separate from the database and either stores or projects the data in a row-oriented format. This means it can perform high-frequency data modifications which are periodically merged with the main columnar-oriented store. The downside of this strategy is that the correct data could be located in either of the stores, and a query might need access to both stores to return an accurate response.

% Projections
In some systems, certain columns are often accessed together. When that is the case, the system might benefit from storing the columns together on disk through a \emph{projection}. A table can then be represented as a series of projections. A projection of all columns in a table is called a \emph{superprojection}. Projections can also be made to increase the performance of certain queries. They can be created by the database administrator, or as needed by the query optimizer.

\subsubsection{Column-Family Stores}

% Components
In a Column-Family Store, the data is stored in column families. Each row in a column family has a key and multiple columns associated with it. Each row in a column family can have a unique set of columns not necessarily the same as any other rows. Although the rows in a column family should be related and will often be accessed together. A column can also consist of a map of columns, which is referred to as a \emph{super column}. When super columns are used to create a column family, this is called a \emph{super column family}. A super column is useful to keep related data together on disk, but if only a part of the data is needed in a query, some computation is wasted in fetching and deserialize data. All column families and super column families are stored in a \emph{keyspace} at the highest level, which represents a database in a relational system.

% Implementation
The most common implementation of a Column-Family Store is \emph{Cassandra}. A database instance -- or \emph{cluster} -- in Cassandra consists of equal nodes than can easily be added to or removed from the cluster. This makes it very scalable and fast, since each node can handle both reads and writes.

% Scalable
A Cassandra system is configurable to achieve the required tradeoff between availability, consistency, and read/write performance. When a keyspace is created, it can be configured how many replicas of each data should be made, i.e. how many nodes will store a redundant copy of every piece of data. This is used to affect the availability of the system, being able to perform operations even if a node is down. A read or write can be considered successful if it has been written to on node, to a majority of the nodes, or to all of the nodes. Reading from or writing to more nodes gives worse performance, but increases the consistency. The consistency can be set with each query, depending on the need for the specific operation.

% Query
Column family stores are often optimized for reading and does not support query languages with a rich feature set. When retrieving or updating data, better performance is acquired by only querying the needed columns. Cassandra has support for its own query language \emph{Cassandra Query Language}, which is similar to SQL but lacks some features such as joins or subqueries.
