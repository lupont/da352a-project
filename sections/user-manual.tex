Following is the user manual for Bithoven.
The manual will cover all of the features in the command line interface.
There are five primary functions in Bithoven, the functions are:
\begin{itemize}
    \item Logging in and signing up
    \item Searching
    \item Listing
    \item Playing
    \item Playlist
\end{itemize}
All of these functions will be described in more detail.

\subsubsection{Terminology}
To understand the user manual, some basic terminology will be provided.
\begin{itemize}
    \item \textbf{piece}: a particular piece of music in abstract, e.g. \emph{Bolero} by Maurice Ravel.
    \item \textbf{content group}: a collection of multiple pieces belonging together, e.g. \emph{Symphony 5} by Ludwig van Beethoven, which consists of four different movements/pieces.
    \item \textbf{composer}: the person who has written a particular piece or content group, e.g. \emph{Maurice Ravel} or \emph{Ludwig van Beethoven} from the examples above.
    \item \textbf{performance}: a specific performance of a piece or content group, i.e. when a piece of music is played. E.g. Bolero by Maurice Ravel was performed by London Symphony Orchestra and Arpad Joo in a recording from 1995.
    \item \textbf{performer}: a person or group who has taken part in a performance of a piece. E.g. \emph{London Symphony Orchestra} and \emph{Arpad Joo} was the performers in the example above. Performers can be further divided into different categories based on their role in the performance.
    \item \textbf{group}: a type of performer consisting of an orchestra, band, or chamber music group, e.g. \emph{London Symphony Orchestra} from the example above.
    \item \textbf{conductor}: a type of performer that is conducting/leading an orchestra or band and are responsible for the musical interpretation, among other things. E.g. \emph{Arpad Joo} in the example above.
    \item \textbf{soloist}: a type of performer that performs a soloistic part, either completely alone or together with a group and/or conductor. E.g. Piano Concerto in A Minor by Edvard Grieg, performed by \emph{Javier Perianes}, BBC Symphonic Orchestra and Sakari Oramo. In this example, Javier Perianes is the soloist, BBC Symphonic Orchestra is a group, and Sakari Oramo is the conductor.
\end{itemize}

\subsubsection{Logging in and signing up}
When Bithoven is started, the user will be asked to log in.
This is done by entering an email address and a password.
If a new email is entered, a new user account will be created and the user are asked to enter a password.

\subsubsection{Searching}
A common use case in Bithoven is to perform searches.
A search starts with the keyword \texttt{search}, followed by a search category, and finally a search query.
The search results will be listed together with a database id in square brackets, e.g. \texttt{[135]}.
This id can be used for other functions in the program, such as listing, playing, and adding to a playlist.
For some search categories, additional filtering can be done through the keyword \texttt{AND} and an additional query.
Which search categories support this and what the additional filter is, is explained in more detail below.

There are seven different search categories in Bithoven:
\begin{itemize}
    \item \texttt{composer}
    \item \texttt{content-group}
    \item \texttt{piece}
    \item \texttt{performance}
    \item \texttt{performer}
    \item \texttt{album}
    \item \texttt{all}
\end{itemize}

\paragraph{composer}
The composer search category will list all composers in the database whose name contains the provided search query.
In addition to the composer's name, the id of the composer will also be provided.

\paragraph{content-group}
The content group search category will list all content groups whose name contains the provided search query.
All of the pieces in the content group will also be listed, as well as the composer of content group.
The content group, the composer, and all of the pieces will be listed together with their database id.

\paragraph{piece}
The piece search category will list all pieces whose name -- or eventual content group's name -- contains the provided search query.
If a piece is part of a content group, the name of the content group will be listed before the name of the piece.
Additional filtering can be done by using the \texttt{AND} keyword in the search query.
The subquery after this keyword will be used to filter all pieces whose composer's name contains the provided search query.
For example, the command \texttt{search piece Symphony no. 5 AND Beethoven} will list all of the pieces
where the name of the piece, or its content group, contains "Symphony no. 5" and it's composer's name
contains "Beethoven".
The database id of the piece and composer will also be provided.

\paragraph{performance}
The performance search category will list all performances of a piece whose name -- or eventual content group's name -- contains the provided search query.
If a piece is part of a content group, the name of the content group will also be listed before the name of the piece.
The composer of the piece or content group will be listed for each performance.
Additional filtering can be done by using the \texttt{AND} keyword in the search query.
The query after this keyword will be used to filter all performances of a piece whose composer's name contains the provided search query.
A list of all performers, as well as their performer type (group/conductor/soloist) will be listed as well.
The database id of the performance, the piece, the composer, and all of the performers will also be provided.

\paragraph{performer}
The performer category will list all performers whose name contains the provided search query.
Note that the performer type will not be listed since this is specific to each performance.
The database id of the performer will also be provided.

\paragraph{album}
The album category will list all albums whose name contains the provided search query.
The database id of the album will also be provided.

\paragraph{all}
The all search category will perform searches of all the different search categories according to
the rules above and will list the results grouped by search category.

\subsubsection{Listing}
Listing can be used to list everything related to a certain entity.
The information displayed is different depending on the type of the entity.
A listing query begins with the keyword \texttt{list}, and is followed by a database id.
Information can be listed for each of the following types:
\begin{itemize}
    \item composer
    \item content-group
    \item piece
    \item performer
    \item performance
    \item album
\end{itemize}

The type is determined automatically based on the provided id.
If an id of a performer type is provided, this can be further specified by providing one of the
following keywords after the id:

\begin{itemize}
    \item group
    \item conductor
    \item soloist
\end{itemize}
If the provided id is not of a performer type, anything entered after the id will be ignored.

An example of a list command is \texttt{list 62} or \texttt{list 65 conductor}.

\paragraph{composer}
Listing a composer will display all of the content groups and pieces written by the composer.
The pieces are listed first, followed by each content group, with it's pieces listed indented below it.
The pieces are ordered in alphabetical order based on the name, followed by the content groups, also ordered in alphabetical order.
Each piece in a content group will be ordered by it's internal ordering.
The database id of each content group and piece will also be provided.

\paragraph{content-group}
Listing a content-group will display it's composer and all pieces it consists of.
It will also list all performances of the content-group or any of it's pieces.
Each performance will display all performer's name as well as their performance type.
The database id of each performance, performer, piece, and composer will also be provided.

\paragraph{piece}
Listing a piece will display it's composer and it's eventual content group.
It will also list all performances of the piece together with the performance's performers.
The performances will be ordered by it's recording date.
The database id of each performance, performer, and composer will also be provided.

\paragraph{performer}
Listing a performer will display all of the performances the performer has performed on.
The performances will be listed together with it's piece, and composer.
The database id of each performance, piece, and composer will also be provided.

\paragraph{group}
Listing a group will provide the same information as \texttt{performer}, but only where the
specified performer has performed as a group, i.e. not as a soloist or conductor.

\paragraph{conductor}
Listing a conductor will provide the same information as \texttt{performer}, but only where the
specified performer has performed as a group, i.e. not as a soloist or group.

\paragraph{soloist}
Listing a soloist will provide the same information as \texttt{performer}, but only where the
specified performer has performed as a group, i.e. not as a group or conductor.

\paragraph{performance}
Listing a performance will display it's composer, piece and eventual content group, as well as all
of it's performers, consisting of their name and performer type.
The database id of the composer, piece, and all performers will also be provided.

\paragraph{album}
Listing an album will display all the performances on it, which consists of it's composer, piece,
eventual content group, and every performer.
The database id of each performance, composer, piece, and performer will also be listed.

\subsubsection{Playing}
Playing are not supported inside the terminal based UI, but will instead open a URL in the default
browser. All performances will have a URL to a recording available on Spotify, so a Spotify account
is required to use this feature.

The play function is invoked by the keyword \texttt{play} followed by an id to a performance.
Other types are not supported in this application.
% Could open an album aswell. Maybe a playlist, but the user have to start every new performance?

\subsubsection{Playlists}
Playlists are associated with the user that is currently logged in.
A playlist can hold a list of performances that the user wants to group together.
All actions on playlists starts with the keyword \texttt{playlist} and is followed by one of the
following keywords:
\begin{itemize}
    \item \texttt{list}
    \item \texttt{create}
    \item \texttt{delete}
    \item \texttt{rename}
    \item \texttt{add}
    \item \texttt{remove}
\end{itemize} 

\paragraph{list}
Listing can be done in two ways. Either a user can display all of it's playlists by just entering
\texttt{playlist list}. Or a user can display the content of a particular playlist by also entering
the playlist id, e.g. \texttt{playlist list 53}. A playlist can only be listed if it is owned by the
user currently logged in.

Playlists will be listed with their name, number of performances added to them, and their database
id.
The performances in a specific playlist will be listed with their piece name, eventual content group
name, composer, performers as well as the database id for each entity, including the performance.

\paragraph{create}
A user can create a new empty playlist with the keyword \texttt{create}. This must be followed by
the name of the playlist, e.g. \texttt{playlist create Barock'n'Roll}. The name can be of arbitrary
length and consist of multiple words, but should only contain ascii characters. A user can have
several playlists with the same name.

\paragraph{delete}
A user can delete a playlist it owns with the keyword \texttt{delete}, followed by the id of the
playlist, e.g. \texttt{playlist delete 53}.
% Warning if not empty?

\paragraph{rename}
A user can rename any of it's playlists with the keyword \texttt{rename}. This must be followed by
the id of the playlist to rename as well as the new name. E.g. \texttt{playlist rename 53 Playliszt}

\paragraph{add}
A user can add a performance to any of it's playlists with the keyword \texttt{add}, followed by the
id of the playlist to add to and the id of the performance to add, e.g. \texttt{playlist add 35 53}.
The performance will be added last in the playlist. The same performance can be added multiple times
to the same playlist.

\paragraph{remove}
A user can remove a performance from any of it's playlists with the keyword \texttt{remove}, followed
by the id of the playlist to remove from and the id of the performance to remove, e.g.
\texttt{playlist remove 35 53}. If the same performance is added multiple times to a playlist, the
first one is removed.

\subsubsection{Help}
Help can be provided in the application by entering the keyword \texttt{help}.

\subsubsection{Quitting}
The application is quit by first logging out, using any of the keywords: \texttt{logout}, \texttt{quit},
or \texttt{exit}. During the log out screen, an \texttt{EOF} character can be entered to exit the
application. This is \texttt{\^D} in a Unix based system or \texttt{\^Z} on a Windows system.

